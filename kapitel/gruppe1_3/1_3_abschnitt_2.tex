\section{Umsetzung der Trends in den betrachteten Hochschulen}
Im Folgenden soll aufgezeigt werden, auf welche Weise die betrachteten Hochschulen die Trends im Informationsmanagement an Hochschulen umsetzen. Hierbei wird sich auf die Punkte Zentralisierung / Integration, Standardisierung / Serviceorientierte Architektur (SOA), Nutzerorientierung / Serviceorientierung, CIO-Konzept sowie ITIL konzentriert.

\subsection{Zentralisierung / Integration}
\label{subsection_zentralisierung_integration}
Alle betrachteten Hochschulen integrieren mehrere Bestandteile unter einer (oft neu gegründeten) Dachorganisation. Typische Bestandteile dieser Dachorganisation sind das Rechenzentrum, die Bibliothek und die Verwaltung.
An der WWU Münster ist der IKM-Service (Information, Kommunikation und Medien) diese Dachorganisation.\footnote{\url{http://www.uni-muenster.de/Rektorat/ikm/index.html}}

Dieser bestand bereits vor dem Projekt MIRO\footnote{\cite[8]{vogl_bericht_2013}} und wurde als Rahmen für dieses verwendet.\footnote{\cite[47]{bode_informationsmanagement_2010}}

Der IKM-Service besteht konkret aus dem Zentrum für Informationsverarbeitung (ZIV), der Universitäts- und Landesbibliothek Münster (ULB) und der Universitätsverwaltung (UniV). Er „bündelt die an der WWU vorhandenen Kompetenzen im Bereich Informationsbereitstellung und –verarbeitung in einem virtuellen Verbund mit kooperativer Leitung“.\footnote{\url{http://www.uni-muenster.de/Rektorat/ikm/index.html}}

Der IKM-Lenkungsausschuss, der sich „aus den Leitungen der beteiligten Einrichtungen sowie dem Prorektor für strategische Planung und Qualitätssicherung“ zusammensetzt, koordiniert die Zusammenarbeit der Bereiche.\footnote{\url{http://www.uni-muenster.de/Rektorat/ikm/index.html}}

Das ITMC an der TU Dortmund ist unter den betrachteten Dachorganisationen die am wenigsten breit aufgestellte und besteht aus dem Hochschulrechenzentrum und dem Medienzentrum.\footnote{\url{http://www.itmc.uni-dortmund.de/beritmc/ueber-itmc.html}}

Das MICK am KIT setzt sich aus dem Rechenzentrum, der Universitätsbibliothek, den Medieneinrichtungen und der Verwaltung zusammen.\footnote{\url{https://kim.cio.kit.edu/downloads/KIM_UniKaTH061.pdf}} Die Aufgabe des MICK ist es, „umzusetzen, was der Ausschuss für Informationsversorgung empfiehlt“.\footnote{\url{https://kim.cio.kit.edu/downloads/KIM_UniKaTH061.pdf}}

Das kiz an der Universität Ulm integriert IT-Dienste, Medien-Dienste und Bibliotheks-Dienste unter einer gemeinsamen Leitung.\footnote{\url{https://www.uni-ulm.de/einrichtungen/kiz/wir-ueber-uns.html}} Außerdem war „der EDV-Betrieb der Verwaltung schon immer im Universitätsrechenzentrum und nicht in einer eigenen EDV-Abteilung angesiedelt. Dieser Aufgabenbereich wurde nach der Auflösung des Rechenzentrums vom kiz übernommen.“\footnote{\url{https://www.uni-ulm.de/einrichtungen/kiz/it/dienste-fuer-die-verwaltung.html}}

\subsection{Standardisierung / SOA}
In Münster wurde im Zuge von Projekt MIRO eine „einheitliche Architektur innerhalb der IT-Komponenten“ angestrebt und als ein „Ansatz zur Erreichung dieses Zieles“ eine „Serviceorientierte Architektur (SOA)“\footnote{\cite[51]{bode_informationsmanagement_2010}} umgesetzt. Als Gründe für die Einführung einer SOA werden Flexibilisierung, Kostenreduktion und Erhöhung der Wiederverwendbarkeit von IT-Prozessen\footnote{\cite[51]{bode_informationsmanagement_2010}} genannt. Technisch wird die Informationsinfrastruktur über Server- und Storage-Virtualisierung umgesetzt, da hierdurch eine „flexible und kurzfristige Provisionierung von Komponenten“\footnote{\cite[52]{bode_informationsmanagement_2010}} ermöglicht wird.
\newpage
Das Projekt MIRO befasst sich in erster Linie mit der „Schaffung einer Infrastruktur für die Nutzung und Verwaltung von (Web-) Services“\footnote{\cite[52]{bode_informationsmanagement_2010}}, es werden jedoch „generell jede Art von Web-Procedure-Calls (HTTP-Aufrufe, REST-Services etc.) unterstützt.“\footnote{\cite[52]{bode_informationsmanagement_2010}}

In Karlsruhe ist ein Fokus des Projektes KIM die „technologische Umsetzung einer integrierten Service Orientierten Architektur (iSOA). Hierbei handelt es sich um eine auf Webservices basierende Softwaretechnologie zur Realisierung von Dienstleistungen, bei der die Geschäftsprozesse im Vordergrund stehen“.\footnote{\url{http://kim.cio.kit.edu/164.php}}
Wie auch in Münster steht im Vordergrund, durch flexiblere IT-Strukturen die Kosteneffizienz und Transparenz zu erhöhen und zu einer Beschleunigung der Bearbeitungsprozesse zu führen.\footnote{\url{http://kim.cio.kit.edu/164.php}}

Ein explizites Ziel der Serviceorientierten Architektur ist, dass die „heterogene IT-Landschaft der Fakultäten und Einrichtungen [...] erhalten bleiben und durch einen auf der Web Service Architecture (WSA) basierenden Ansatz zu einem homogenen und hochflexiblen Ganzen zusammengefügt werden“\footnote{\url{http://kim.cio.kit.edu/164.php}} kann.

\subsection{Nutzerorientierung / Serviceorientierung}
Eines der obersten Prinzipien bei der Umsetzung des Projekt MIRO an der WWU Münster war „von Beginn an die konsequente Ausrichtung der Dienstleistungen am Bedarf der Nutzer.“\footnote{\cite[19]{vogl_bericht_2013}}

Auf diese ausdrückliche Nutzerorientierung führt die Universität auch zurück, dass „ein so umfassendes Projekt wie MIRO bereits von Beginn an wesentliche Ergebnisse generieren konnte und nicht nur auf dem Campus der WWU Anerkennung erzielte.“\footnote{\cite[19]{vogl_bericht_2013}}

Hierfür wurden u.a. „Ergebnisse ausgewählter Umfragen speziell unter dem Aspekt Informationsverhalten und -bedarf analysiert“\footnote{\cite[19]{vogl_bericht_2013}} sowie „Bedarfsanalysengespräche mit Wissenschaftlern unterschiedlicher Fachbereiche und Institute geführt“ um „den Status quo im Umgang mit wissenschaftlichen und organisatorischen Informationen [...] zu erfassen“\footnote{\cite[19]{vogl_bericht_2013}} und „Bedarfe und Verbesserungspotentiale aufzuspüren.“\footnote{\cite[20]{vogl_bericht_2013}}

Der Beirat des ITMC an der TU Dortmund wurde explizit eingerichtet, um „Nutzerorientierung zu gewährleisten“\footnote{\url{http://www.itmc.uni-dortmund.de/beritmc/ueber-itmc/beirat-des-itmc.html}}. Als „zentrale Anlaufstelle für alle Fragen rund um die Dienstleistungen des ITMC“\footnote{\url{http://www.itmc.uni-dortmund.de/dienste/support-weiterbildung/service-desk.html}} gibt es den Service Desk, der einen umfangreichen Dienstleistungskatalog bereitstellt.\footnote{\url{http://www.itmc.uni-dortmund.de/component/phocadownload/category/158-ordnungen-und-regelungen.html?download=637:dienstleistungskatalog}}

Dort sind auch die Service Level definiert, wobei  die Systeme 24x7 (ausgenommen definierte Zeitfenster für Wartungsarbeiten) und der Support 8x5 zur Verfügung stehen.\footnote{\url{http://www.itmc.uni-dortmund.de/component/phocadownload/category/158-ordnungen-und-regelungen.html?download=637:dienstleistungskatalog, Seite 7}}

Für das Projekt KIM-CM (KIM Campus Management), einem Teilprojekt von Projekt KIM, in Karlsruhe gehörte es zu den zentralen Projektgrundsätzen, „alle Anspruchsgruppen im Rahmen von Facharbeitsgruppen und dem Studierenden-Arbeitskreis in das Projekt“\footnote{\url{http://kim.cio.kit.edu/516.php}} einzubeziehen, „so dass die neue Software bestmöglich an den Bedürfnissen aller Anwender und Nutzer ausgerichtet wird. Das Projekt KIM-CM zielt auf eine Optimierung aller Geschäftsabläufe, so dass alle betroffenen Gruppen davon profitieren.“\footnote{\url{http://kim.cio.kit.edu/516.php}}

\subsection{Das CIO-Konzept}
\label{cio_konzept}
In Münster gibt es keine Einzelperson als CIO. Stattdessen „wurde als Steuerungsgremium der IV-Lenkungsausschuss (IV-L) initiiert“\footnote{\cite[59]{bode_informationsmanagement_2010}}, der „direkt dem Rektorat zugeordnet“\footnote{\cite[59]{bode_informationsmanagement_2010}} ist. Die Aufgaben des IV-L sind „u.a. die Sicherung des nutzergerechten und wirtschaftlichen Betriebs des Gesamtsystems und die Festlegung sowie Kontrolle von Zielen und Aufgaben auf zentraler und dezentraler Ebene.“\footnote{\cite[9]{vogl_bericht_2013}}

Damit ist der IV-L „dem CIO von Unternehmen vergleichbar, dabei allerdings gut an die Gegebenheiten der Universität angepasst.“\footnote{\cite[60]{bode_informationsmanagement_2010}}

Der IV-L setzt sich zusammen aus dem Rektor/der Rektorin oder einem Prorektor/einer Prorektorin, dem Kanzler/der Kanzlerin, dem oder der Vorsitzenden der IV-Kommission, der Leiterin oder dem Leiter des IV-Zentrums sowie der Leiterin oder dem Leiter der ULB, sowie drei weiteren Mitgliedern und deren Stellvertreterinnen und Stellvertreter.\footnote{\url{http://www.uni-muenster.de/wwu/leitung/ausschuesse/iv-lenkung.shtml}}
Er trifft sich zweimal pro Semester.\footnote{\url{http://www.uni-muenster.de/wwu/leitung/ausschuesse/iv-lenkung.shtml}}

An der TU Dortmund erfüllt der Leiter des ITMC die Funktion des CIO.\footnote{\url{http://www.tu-dortmund.de/uni/Uni/Zahlen__Daten__Fakten/Statistik/Publikationen/Jahrbuch/Jahrbuch_2009__kl.pdf, Seite 37}}
Außerdem gibt es den Beirat des ITMC, der mindestens zweimal pro Jahr tagt und Stellung zu dem Entwicklungskonzept des ITMC, der Budgetplanung für das ITMC, dem Dienstleistungskatalog, der Zielvereinbarung und dem Jahresbericht nimmt.\footnote{\url{http://www.itmc.uni-dortmund.de/beritmc/ueber-itmc/beirat-des-itmc.html}}

Auch in Karlsruhe gibt es einen CIO. Dieser „ist KIT-weit für die technische, organisatorische und nutzungsrechtliche Integration und Koordination aller Aktivitäten in den Bereichen Information und Kommunikation zuständig.“\footnote{\url{http://www.kit.edu/cio/index.php}}

In Ulm gibt  es die Position des CIO nicht, aufgrund der starken Integration der unterschiedlichen Informationsdienste kann aber wohl davon ausgegangen werden, dass die Leitung des kiz einen Großteil der Aufgaben übernimmt, die in den Aufgabenbereich eines CIO fallen würden.

\subsection{ITIL}
Die IT Infrastructure Library (ITIL) findet zwar häufig Erwähnung, nimmt jedoch in der praktischen Umsetzung des Informationsmanagements an den betrachteten Hochschulen keine wichtige Rolle ein.

An der WWU Münster war es eines der Ziele von Projekt MIRO, projektbegleitend „verschiedene Dienstleistungen zu vervollständigen und zu verbessern. Das betrifft die Themen Sicherheit, System- und Netzwerkmanagement, die Einführung von Service-Levels für angebotene Dienste und eine deutlichere Strukturierung der Dienste im Sinne von ITIL (IT Infrastructure Library).“\footnote{\url{http://www.ulb.uni-muenster.de/bibliothek/aktivitaeten/projekte/projekt_miro.html}}.

An der TU Dortmund wurde 2008 für den Service Desk aus Mitteln des Landes NRW Software „mit angepassten ITIL-konformen Frameworks“\footnote{\url{http://www.itmc.uni-dortmund.de/beritmc/dokumente/itm-update/841-nr5-servicedesk.html}} beschafft.
An der Universität Karlsruhe wurde 2007 ein Pilotprojekt durchgeführt.\footnote{\url{https://indico.cern.ch/event/18714/session/32/contribution/144/material/slides/1.pdf}}
In Ulm ist ITIL von den betrachteten Universitäten am stärksten im Einsatz. Eine der Aufgaben der erst 2014 gegründeten\footnote{\url{http://www.uni-ulm.de/index.php?id=57466}} Abteilung Servicemanagement und Organisation des kiz umfasst „Modellierung und Management von Service-Prozessen (insb. nach ITIL-Standard)“.\footnote{\url{http://www.uni-ulm.de/index.php?id=57466}}
