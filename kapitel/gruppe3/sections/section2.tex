\section{Support und Fortentwicklung}
Support und Fortentwicklung hängen hier eng zusammen, da die Fortentwicklung hauptsächlich durch die im Support gewonnenen Einsichten über Defizite in Prozessen vorangetrieben werden soll. So sollen Diskrepanzen zwischen Erwartungen an das System und dessen tatsächlichen Fähigkeiten und Nutzung aufgedeckt und behoben werden.

\subsection{Support}
Supportleistung an einer kleinen Hochschule geschieht häufig direkt und unbürokratisch. Dieser ad-hoc-Ansatz bringt zwar vielfach schnelle Hilfe, aber nur wenig zuverlässige Informationen über Prozessdefizite.

\subsubsection{Zentrale Dokumentation}
Die Vorteile dieser Art der Hilfeleistung sind für eine kleine Hochschule allerdings evident. Der Overhead mehr reglementierter Supportsysteme würde einen unverhältnismäßigen Personalaufwand mit sich bringen, und Hilfeleistung verzögern. Die Qualität des Supportprozesses selber würde damit sinken.
Notwendig zur besseren Identifizierung von Prozessdefiziten ist allerdings keine zentralisierte Supportleistung an sich, sondern lediglich eine zentralisierte Dokumentation des geleisteten Supports.

\begin{figure}[h!]
	\centering
	\includegraphics[width=10cm]{kapitel/gruppe3/bilder/grafik_supportlog}
	\caption{Unabhängige Supportleister dokumentieren in zentralem Log}
	\label{fig_zentraler_supportlog}
\end{figure}

Es ist dabei unerheblich, ob die Supportleister die Unterstützung als Kern ihrer Aufgabe leisten, oder ob es sich um kollegiale Unterstützung bei einem Problem handelt. Gerade letztere Information aufzufangen ist wichtig, da diese sonst nur eine sehr schwer einzuschätzende Größe bleibt.

Dieser Dokumentationsoverhead ist gering gegenüber dem Overhead eines stark reglementierten Supportsystems, erhält alle Vorteile unbürokratischer, schneller Hilfeleistung und fängt zusätzlich Informationen über Art und Umfang gelisteten Supports auf.

\subsubsection{Knowledge Base}
Aus dem Supportlog kann eine durchsuchbare Knowledge Base aufgebaut werden, die nicht nur die allgemeinen Fehlerquellen und Schwierigkeiten von Software im Einsatz beleuchtet, sondern ganz speziell die an der Hochschule in dieser Zusammenstellung einmalige Konfiguration.
Dadurch kann sehr viel schneller auf spezifische Fehlerszenarien reagiert werden, als dies mit allgemeinen Informationen möglich ist.

Auch können aus dem Supportlog FAQs abgeleitet werden, die tatsächlich dem Wortsinn nach Listen häufig gestellter Fragen und Antworten darstellen, und nicht was mehr oder minder begründet vermutet wird.

Auch zeigt sich in den Häufigkeiten bestimmter Probleme, wo spezielle Dokumentation und Hilfetexte notwendig sind, die ebenfalls hinterlegt werden können.

Hierzu muss das Supportlog allerdings von einer geeigneten Stelle regelmäßig gesichtet werden.

\subsection{Fortentwicklung}
Eine Konzeption kann nur aktuelle Trends und Entwicklungen berücksichtigen. Es ist schwierig vorauszuschauen, was die Zukunft danach bringen wird, welche Trends mehr oder weniger wichtig sind, und welche Trends darauf folgen werden.

Allerdings ist es keine Frage, dass eine Hochschule länger Bestand hat, und es damit sinnvoll ist, Prozesse zu hinterlegen, die neue Trends und Entwicklungen zwar nicht vorwegnehmen, aber deren zeitnahe Entdeckung und Integration ermöglichen.

Auch zeigt sich in der Praxis, dass unvorhergesehene Bedingungen und Ereignisse theoretisch gut ausgearbeitete Prozesse übermäßig blockieren, und eine Anpassung geschehen muss.

\subsubsection{Feedback}
\label{feedback}
Hierzu muss an jedem Punkt des Gesamtsystems dem Benutzer möglich sein, Feedback zu geben. Mehr noch muss gerade bei neuen oder überarbeiteten Prozessen dieses Feedback eingefordert werden, um die Qualität des neuen Prozesses oder Tools einschätzen zu können.

Das Feedback gelangt an die zuständige Stelle, muss aber auch zentral gesammelt werden, ähnlich wie das Supportlog. Diese Sammlung wird zentral ausgewertet, um verdeckte, verteilte Probleme aufzudecken, die sich in Feedback an unterschiedliche Stellen verbergen können.

Auf die Auswertung muss wo sich Probleme zeigen, eine Information der zuständigen Stelle folgen, damit eine Verbesserung erarbeitet werden kann. Entsprechend ist die zuständige Stelle berechtigt, ein Meeting einzuberufen, damit ihre Eingaben nicht einfach verloren gehen können, sondern zwangsläufig mindestens einmal besprochen werden.

\subsubsection{Innovationseingabe}
In den Feedbackprozess eingebettet muss die Möglichkeit für jede Person sein, Innovationen aus beliebiger Quelle zu beschreiben, so dass Entwicklungen nicht erst von bestimmter Stelle wahrgenommen werden müssen, um erwägt zu werden. Damit kann von beliebiger Stelle aus eine Verbesserung zur Diskussion gebracht werden.

Damit diese Möglichkeit von Benutzern angenommen wird, muss auf Eingaben angemessen schnell reagiert werden. Um eine ernsthafte Reaktion zu gewährleisten, müssen diese Vorschläge auch diskutiert worden sein.

\subsubsection{Erfahrungsgetriebene Fortentwicklung}
Aus den Erkenntnissen über Schwachstellen aus dem Supportlog, den Benutzerberichten und -bewertungen aus dem Feedbacklog und den Innovationseingaben können nicht nur Schwachstellen und Fehler in Prozessen identifiziert werden, sondern auch Trends in der Benutzung des Systems erkannt. Da Support und Feedback andauernde Prozesse sind, ergibt sich daraus ein selbstregulierendes System, das, wenn die Messgrößen Supportlog und Feedbacklog angemessen berücksichtigt werden, evolutionär verbessert.

\begin{figure}[h!]
	\centering
	\includegraphics[width=10cm]{kapitel/gruppe3/bilder/zyklus_prozessverbesserung}
	\caption{Zyklus der Verbesserung eines Prozesses}
	\label{fig_zyklus_prozessverbesserung}
\end{figure}





















